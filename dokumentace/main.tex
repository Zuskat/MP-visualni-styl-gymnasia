\documentclass[11pt,a4paper,twoside,openright]{report}

\usepackage[top=25mm,bottom=25mm,right=25mm,left=30mm,head=12.5mm,foot=12.5mm]{geometry}
\let\openright=\cleardoublepage

\usepackage[a-2u]{pdfx}

%% Přepneme na českou sazbu, fonty Times New Roman a kódování češtiny
\usepackage[czech]{babel}
\usepackage{times}
\usepackage[T1]{fontenc}
\usepackage{textcomp}
\usepackage[utf8]{inputenc}

%%% Další užitečné balíčky (jsou součástí běžných distribucí LaTeXu)
\usepackage{graphicx}       % vkládání obrázků
\usepackage{dcolumn}        % lepší zarovnání sloupců v tabulkách
\usepackage{booktabs}       % lepší vodorovné linky v tabulkách
\makeatletter
\@ifpackageloaded{xcolor}{
   \@ifpackagewith{xcolor}{usenames}{}{\PassOptionsToPackage{usenames}{xcolor}}
  }{\usepackage[usenames]{xcolor}} % barevná sazba
\makeatother
\usepackage{multicol}       % práce s více sloupci na stránce
\usepackage{caption}
\usepackage{enumitem}
\usepackage{lipsum}
\setlist[itemize]{noitemsep, topsep=0pt, partopsep=0pt}
\setlist[enumerate]{noitemsep, topsep=0pt, partopsep=0pt}
\setlist[description]{noitemsep, topsep=0pt, partopsep=0pt}
\usepackage{pdfpages}

% vyznaceni odstavcu
\parindent=0pt
\parskip=11pt

% zakaz vdov a sirotku - jednoradkovych pocatku ci koncu odstavcu na prechodu mezi strankami
\clubpenalty=1000
\widowpenalty=1000
\displaywidowpenalty=1000

% nastaveni radkovani
\renewcommand{\baselinestretch}{1.2}

% Zapne černé "slimáky" na koncích řádků, které přetekly, abychom si
% jich lépe všimli.
\overfullrule=1mm



\def\NazevPrace{Grafický visuální styl gymnasia}
\def\Trida{4.B}
\def\AutorPrace{Kateřina Zusková}
\def\DatumOdevzdani{2023}

% Vedoucí práce: Jméno a příjmení s~tituly
\def\Vedouci{Šimon Schierreich}

% Studijní program a obor
\def\StudijniProgram{studijní program}
\def\StudijniObor{studijní obor}

% Text čestného prohlášení
\def\Prohlaseni{Prohlašuji, že jsem svou práci vypracoval samostatně a použil jsem pouze prameny a literaturu
uvedené v~seznamu bibliografických záznamů. Nemám žádné námitky proti zpřístupňování této práce v~souladu se
zákonem č. 121/2000 Sb. o~právu autorském, o~právech souvisejících s~právem autorským a
o~změně některých zákonů (autorský zákon) ve znění pozdějších předpisů.}

% Text poděkování
\def\Podekovani{%
Poděkování.
}

% Abstrakt česky
\def\Abstrakt{%
Práce se zabývá tvorbou visuální identity Gymnázia Jana Keplera, sestavením grafického manuálu k její aplikaci a přípravou šablon usnadňujících její použití.
}

% Abstrakt anglicky
\def\AbstraktEN{%
Abstract.
}

% 3 až 5 klíčových slov
\def\KlicovaSlova{visuální identita, grafický manuál, šablona}
% 3 až 5 klíčových slov anglicky
\def\KlicovaSlovaEN{keyword, important term, another topic, and another one}


\begin{document}

%%% Titulní strana práce a další povinné informační strany

%%% Titulní strana práce

\pagestyle{empty}
\pagenumbering{gobble}
\hypersetup{pageanchor=false}

\begin{center}
\LARGE
\textbf{GYMNÁZIUM JANA KEPLERA}\\
{\large Parléřova 2/118, 169 00 Praha 6}

\vspace{\stretch{3}}

\includegraphics[width=.4\textwidth]{img/logo}

\vspace{\stretch{3}}

{\Huge\bfseries\NazevPrace}

\vspace{8mm} 
Maturitní práce

\vspace{\stretch{8}}
\large
\begin{tabular}{rl}
Autor: & \AutorPrace \\
\noalign{\vspace{2mm}}
Třída: & \Trida\\
\noalign{\vspace{2mm}}
Školní rok: & \SkolniRok\\
\noalign{\vspace{2mm}}
Předmět: & \Predmet \\
\noalign{\vspace{2mm}}
Vedoucí práce: & \Vedouci \\
\end{tabular}

\vspace{20mm}
Praha, \DatumOdevzdani
\end{center}


\openright


% zadani viz šablona zadání

\includepdf[]{sablona_zadani.pdf}

%%% Poděkování
\openright
\vspace*{\fill}
\section*{Poděkování}
\noindent
\Podekovani
\vspace{1cm}


%%% Strana s čestným prohlášením k bakalářské práci

\hypersetup{pageanchor=true}
\cleardoublepage
\vspace*{\fill}
\section*{Prohlášení}
\noindent
\Prohlaseni

\vspace{2cm}
\noindent
V Praze dne \today
\hspace*{\fill}\small{\AutorPrace}
\vspace{1cm}


%%% Povinná informační strana bakalářské práce
\openright
\section*{Abstrakt}
\noindent
\Abstrakt
\subsection*{Klíčová slova}
\noindent
\KlicovaSlova

\vfill

\section*{Abstract}
\noindent
\AbstraktEN
\subsection*{Keywords}
\noindent
\KlicovaSlovaEN

\openright
\pagenumbering{arabic}

% Obsah
\setcounter{tocdepth}{2}
\tableofcontents

\chapter{Teoretická část}
\pagestyle{fancy}

V první části maturitní práce by se měla objevit informace o tom, jaký problém řešíte. Co si Váš projekt klade za cíl?

Gymnázium Jana Keplera nemá sestavený žádný grafický manuál. Jeho visuální identita je nejasná a používána nejednotně a bez pravidel. Mým cílem je sestavit grafický manuál, který by měl zabránit nejednotě visuální komunikace, definovat a zpřehlednit jeho visuální identitu tak, aby s ní bylo možné snadno pracovat. Grafický manuál by měl určovat, jak používat logo, barvy, fonty a další grafické prvky. Zároveň je sestaven tak, aby se jím mohl řídit jakýkoliv designér používající jakékoliv médium.
Dále je mým cílem sestavit šablony usnadňující použití visuální identity v nejpoužívanějších médiích pro často užívané dokumenty a další náležitosti a navrhnout úpravy webových stránek tak, aby byly v souladu s grafickým manuálem.

**logoxlogotypxpiktogram**
**v implementaci mám možná info hodící se spíše do teoretické sekce - VYŘEŠIT**

\section{Grafický manuál}
Rozhodla jsem se udělat grafický manuál v pdf. Nabízí se udělat také responzivní grafický manuál v podobě webové stránky. Webová stránka nabízí responzivitu. Nicméně forma pdf je běžnější, pro spoustu lidí srozumitelnější a vhodnější pro potřeby gymnasia. Zároveň ji lze také snadno vytisknout a používat i offline. Vzhledem k tomu, že gymnasium nepotřebuje žádný velký a detailně propracovaný grafický manuál, rozhodla jsem se jej udělat jednoduchý a forma webové stránky by jej zbytečně zesložiťovala. Zároveň by to byla spousta práce navíc, která by téměř neměla smysl.
\section{Šablony}
\subsection{Zadání závěrečné práce}
\subsection{Závěrečná práce}
\subsection{Dopisní papír}
\subsection{Email}
\subsection{Prezentace}
\subsection{vizitky}
\section{Návrhy úprav webových stránek}



\chapter{Implementace}

Druhá kapitola obsahuje detailní informace o tom, jak probíhala implementace. Zde se objeví zdůvodnění výběru technologií, řešení problémů, na které jste narazili, informace o použitých knihovnách apod. Pochvalte se, nikdo to za Vás neudělá. Přiznejte chyby, není to ostuda.

\section{Vektorová grafika}
\label{sec:grafika}
Nedílnou součástí projektu je tvorba grafiky. Ačkoli škola například logo má a mým úkolem není vymýšlet logo nové, je potřeba ho zvektorizovat a navrhnout povolené a zakázané varianty. Zároveň bych ráda navrhla i samotný logotyp a variantu, kdy je logo použito s textem v bloku - na webových stránkách je momentálně text vedle loga, tudíž se text dvakrát opakuje (viz \nameref{fig:logotyp_web})

\begin{figure}[htbp]
  \includegraphics[width=1\textwidth]{img/gjk_web_hlavicka.png}
  \caption{logotyp na webu www.gjk.cz}
  \label{fig:logotyp_web}
\end{figure}

%\includegraphics[width=1\textwidth]{img/gjk_web_hlavicka.png}
%    \caption{logotyp na webu gjk}
%    \label{fig:logotyp_web}

Pro loga, logotypy atp. je nejlepší vektorová grafika, aby bylo možné různě upravovat jejich velikost a podobu bez ztrát na kvalitě.

Pro tvorbu vektorové grafiky se v profesionálním prostředí nejčastěji používá software od Adobe nebo Affinity. Tyto možnosti jsem však zavrhla (viz \nameref{sec:manual}).

Nejznámějšími programy pro tvorbu grafiky zdarma jsou Inkscape, GIMP a Krita. Existuje samozřejmě mnoho dalších, většina mě známých však nemá dostatek nástrojů nebo je ve vývoji.
GIMP je podle definice "multiplatformní nástroj pro manipulaci s fotografiemi“ a používá rastrovou a bitmapovou grafiku. Na tvorbu loga atp. se tedy příliš nehodí.
Inkscape je podle definice „otevřený nástroj pro kreslení pro vytváření a úpravy grafiky SVG,“ používá vektorovou grafiku a jedná se snad o nejlepší alternativu k Adobe Illustrator, proto jsem jej zvolila jako vhodný program.
Krita také umí pracovat s vektrovou grafikou a vytvořit v ní logo je možné. Primárně ale pracuje s rastrovou grafikou a je určená hlavně pro umělce zabývající se concept artem, ilustracemi a komiksy. Inkscape je tedy vhodnější už proto, že se zabývá výhradně vektorovou grafikou a nabízí více možností z hlediska vektorové grafiky. S Inkscapem navíc mám trochu zkušeností a pracovat s Kritou, která je komplexnější, ale zároveň mnohem složitější a uživatelsky méně intuitivní, by pro mě bylo zbytečně mnohem složitější.


https://cisatrochu zkušeností a pracovat s Kritou, která je mnohem design.cz/logo-vs-logotyp/
https://daviesložitějsmediadesign.com/cs/inkscape-vs-gimp-which-one-should-you-use/
https://www.quora.com/How-would-you-compare-Inkscape-vs-Krita-Which-one-is-better-for-website-and-brand-design-including-logos

\section{Grafický manuál}
\label{sec:manual}
K tvorbě grafického manuálu jsem se rozhodla použít sázecí program Scribus. Jedná se o open-source software, který je zdarma. Má grafické rozhraní, díky čemuž bude moct grafický manuál v budoucnosti aktualizovat i člověk, který nemá příliš zkušeností s prací se sázecími programy.

Jako další možnosti jsem zvažovala použití softwaru od Adobe nebo Affinity. Tento software je používán pro tvorbu grafických manuálů v profesionálním prostředí a určitě by nebyl špatnou volbou, co se týče nabízených nástrojů. Rodina Adobe i Affinity navíc nabízí i špičkové programy pro tvorbu vektorové grafiky (více o výběru sw pro tvorbu grafiky v sekci \nameref{sec:grafika}). Bohužel, Adobe i Affinity jsou licencované produkty vyžadující nemalou investici peněz. Nabízí sice zlevněné edukační a studentské licence, ale ani tak se nejedná o zrovna levné programy. Škola by musela platit licenci, aby manuál mohl být dále aktualizován. Ačkoliv je nabízený software profesionální, nabízí spoustu featur, které nejsou pro mé potřeby nutné. Zároveň by se kdokoliv, kdo by manuál dále aktualizoval, musel se softwarem seznámit, což je pro mnohé lidi, kteří se v grafice nepohybují, u složitých programů složité. Narozdíl od Adobe a Affinity byl program Scribus využíván studenty pro sazbu školního časopisu Tycho, program je nainstalovaný na školních počítačích a obecně má na škole větší zázemí. Nástroje Scribusu bohatě stačí pro můj projekt a zároveň je zdarma. Rozhodla jsem se tedy do Adobe/Affinity neinvestovat, vzhledem k tomu, že Scribus je dostatečný.

Jako další možnost se jeví použítí LaTeXu. Ten je vhodný ale spíše pro práci s textem, ne s velkým množstvím grafického obsahu, který je nedílnou součástí grafického manuálu. Další nevýhodou by bylo, že budoucí aktualizace manuálu by mohli dělat pouze lidé, kteří latexu rozumí, kterých na škole není příliš velké množství. Se Scribusem sice také většina lidí běžně nepracuje, nicméně je mnohem intuitivnější.

Dále by bylo možné grafický manuál sázet v online nástrojích jako Canva, nebo dokonce i v PowerPointu. Tyto možnosti jsem ale zavhrla vzhledem k jejich omezeným možnostem a také proto, že jejich použití zkrátka vůbec nepůsobí profesionálně - ačkoliv nejsem profesionál, chci, aby výsledek měl alespoň nějakou úroveň. Navíc oba programy jsou určeny primárně pro jiné účely, tedy i proto nenabízejí tak dobré možnosti, jako Scribus, Adobe a Affinity.

MS Publisher - nemám licenci, nemá to takový layout, nepodporuje to svg a musela bych převádět files, ale umí lépe pracovat s barvama, je jen na win. Je jednodušší a víc user-friendly. 

\section{Šablony}
\section{Návhry úprav webových stránek}


\chapter{Technická dokumentace}

Poslední kapitola obsahuje informace o tom, jak projekt, který v rámci maturitní práce vznikl, nainstalovat, spustit a používat.

\section{Ukázka sekce}

\lipsum[5]

\subsection{A jedné podsekce}

\lipsum

\section{A další sekce}

\lipsum

\chapter*{Závěr}
\pagestyle{empty}
\addcontentsline{toc}{chapter}{Závěr}

Závěr obsahuje shrnutí práce a vyjadřuje se k míře splnění jejího zadání. Dále by se zde mělo objevit sebehodnocení studenta a informace o tom, co nového se naučil a jak vnímal svou práci na projektu.

%%% Seznam použité literatury
\nocite{einstein}\nocite{latexcompanion}\nocite{knuthwebsite}
\printbibliography[title={Seznam použité literatury},heading={bibintoc}]

%%% Seznam obrázků
\openright
\listoffigures
\addcontentsline{toc}{chapter}{Seznam obrázků}

%%% Seznam tabulek
\clearpage
\listoftables
\addcontentsline{toc}{chapter}{Seznam tabulek}

%%% Přílohy k práci, existují-li. Každá příloha musí být alespoň jednou
%%% odkazována z vlastního textu práce. Přílohy se číslují.

%\part*{Přílohy}
%\appendix

\end{document}
